\section*{CONCLUSION}
\addcontentsline{toc}{section}{CONCLUSION}
Managing waste remains a global challenge. The wind farm industry is growing at an increasing rate, and so are wind turbine waste, byproducts and environmental damage. We investigated the most cost-effective way of helping the province of Ontario in Canada of approaching the two key challenges this novel wind turbine recycling industry faces. A recycling facility location problem and specialized recycling vehicle routings. We modelled wind turbine recycling facility locations (fiberglass recycling facilities) in the province of Ontario over a 50-year period to minimize the facility and transportation costs. While simultaneously investigating how many specialized recycling trucks are required and what routes these trucks should take on.   

In the first stage, we used the Capacitated Multi-Facility Weber Problem to find the optimal location of recycling facilities that would minimize the total cost. Both the cost of delivering wind turbines and the fixed cost associated with the construction of a recycling facility make up the total cost. In this stage, our finding suggests that over a 50-year period, the fixed cost of building recycling facilities tends to be lower than the long-term cost of delivering wind turbines to these facilities. Therefore, somewhat counterintuitively, more facilities reduce the overall cost over 50 years. As we spread facilities throughout the province, the lower transportation costs reduce the overall cost. We pushed this location model further to expand across Canadian provinces. When comparing a provincial and a national approach, our results indicate that a provincial approach is cheaper. When each province handles recycling, each province can reduce the distance between the facility and wind farms, which is the main cost driver. Furthermore, each province can install and legislate on its own facilities to fit within its infrastructure which may be more feasible for real-world applications. A Federal Program would serve the entire country, which would maximize the facility capacity but increase transportation costs.  

In the second stage of our approach, with the Recycling Truck Service Rotation problem, the goal was to minimize the number and cost of specialized trucks that travel between wind farms to cut the discarded blades into smaller pieces for easier transportation. To reduce the complexity of the Traveling Salesperson type problem (TSP), we used a KNN clustering algorithm to determine the sub-tours. In other words, each cluster became a tour for each recycling truck. Based on movement and fixed cost, the optimal number is 3. The TSP problem will be further refined with more accurate estimates of capacity, fixed cost, transportation costs and transportation time. This second problem is an important step in long-term wind turbine recycling.  

Humans have a long-standing tendency to create problems as a direct result of solving another problem. We harness fire to create warmth, produce wonderous inventions and harness electricity only to change our environment. We harness the wind to slow the environmental damage only to create more unintended waste. Perhaps this time is different. We have the data, computational power, and an ethical business lens. We see and understand the impact of our actions and know how to fix the situation. We hope the approach to these two key challenges, location, and routing optimizations, enables a swift transition to a more sustainable, green, wind turbine industry. 