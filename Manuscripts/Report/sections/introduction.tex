%%%%%%%%%%%%%%%%
%%% Introduction
%%%%%%%%%%%%%%%%
\section*{INTRODUCTION}
\addcontentsline{toc}{section}{INTRODUCTION}



\indent Fire burns. Wind blows. Humans have mastered fire, but not its consequences. Fire creates CO2, a major contributor to greenhouse gas emissions that permanently alters our environment. Electricity generation from coal and natural gas is a major global contributor to annual CO2 emissions. The pursuit of net zero emissions by 2050 is a continuous challenge \cite{RN19}. Over the last 3 decades, wind turbines have appeared as a leading source of clean, reliable, and renewable electricity generation. More wind farms are being commissioned at an increasing rate. Engineers have begun to master the wind with larger, more efficient, and cheaper turbines \cite{RN18}. As time moves forward net zero emissions in the next thirty years seems an achievable goal. However, wind turbines have a dirty little secret: replaced blades. The wind turbine blades are progressively degraded by the environment and crucially dust \cite{RN1}. At the end of their 20-year lifespan, they need replacing. Most wind turbine blades are made from lightweight and strong fiberglass composites that are difficult to recycle. So, most old blades are buried or left in the fields. The blades eventually degrade contaminating groundwater and soil. We should not have to trade off greenhouse gas emissions and environmental degradation. Therefore, we must recycle the blades to preserve the environment. 

There is good news as engineers have developed ways of recycling fiberglass blades at specialized facilities \cite{RN3}. However, the location of these facilities remains a challenging subject. This is where data scientists can help. Optimizing the locations of facilities is a data challenge. Specifically, we must balance high capital costs for specialized equipment and long-term transportation costs between wind farms and facilities. There may be many optimal solutions or formulations. For instance, is it more cost-effective to build 20 facilities to reduce shipping costs or only 1 facility with increased shipping costs but low capital costs? Further increasing the complexity, the facilities must maintain enough supply of discarded blades to continue working or risk standing idle. Wind farms will replace all the blades at once creating a lumpy supply that creates additional challenges. We will model and address these constraints through data science and the art of optimization. 

We will examine the wind turbine industry of Ontario, considering a planning horizon of 50 years. Ontario produces around 40\% of all wind energy in Canada \cite{RN10}.  Therefore, Ontario has sufficiently diverse wind farm locations and a large enough dataset to fully model the complexity of the problem. The challenge, here, is to perfect the locations for these facilities within Ontario such that costs are minimized while ensuring a sufficient continuous supply of blades for processing. To push the problem further we will investigate a travelling salesman-style problem, involving specialized blade-cutting machinery. This equipment must travel to the wind turbines to cut them up for easier transportation. The recycling costs are associated with the cutting, transportation, storage, and disposal of the blades. Wind turbine blades have a lifetime of 20 years \cite{RN2}.